% Section 3 - Docker Compose
% Roberto Masocco <roberto.masocco@uniroma2.it>
% June 4, 2024

% ### Docker Compose ###
\section{Docker Compose}
\graphicspath{{figs/section3/}}

% --- Composing services ---
\begin{frame}{Composing services}
	\begin{columns}
		\column{.5\textwidth}
		Managing multiple, interdependent \textbg{containerized services} can become quite a tedious task.\\
    \bigskip
		Each container may take multiple options, some have to be started in sequence or built in a particular way...
		\begin{block}{}
			\centering
			\textbf{Compose} is a utility that helps to \textbf{build}, \textbf{run} and \textbf{manage} multiplatform containers by parsing all such settings from \textbf{YAML configuration files}.
		\end{block}
    For instance, \textbg{our code repository} cointains development containers managed by Compose.

		\column{.5\textwidth}
		\begin{figure}
			\centering
			\includegraphics[scale=.25]{composeLogo.png}
			\label{fig:compose}
			\caption{Docker Compose logo.}
		\end{figure}
	\end{columns}
\end{frame}

% --- Compose files ---
\begin{frame}[fragile]{Compose files}
	\begin{columns}\column{.9\textwidth}
		\begin{lstlisting}[language=compose, caption=Minimal example of a Compose file.]
services:
  development:
    build:
      context: .
      args:
        TARGET: dev
    image: devenv:latest
    environment:
      TERM: xterm-256color
    network_mode: host
    command: ["/bin/zsh"]
    volumes:
      - ~/.ssh:/home/user/.ssh
\end{lstlisting}
	\end{columns}
	Refer to the \href{https://docs.docker.com/compose/compose-file/}{\color{blue}\underline{Compose reference}} for more.
\end{frame}

% --- Compose commands ---
\begin{frame}{Compose commands}
	Pretty much the same that Docker has, but invoked with\\
  \bigskip
	\texttt{docker compose}\\
  \bigskip
	instead of \texttt{docker} (previously \texttt{docker-compose}), and oriented only towards services specified in the local Compose file.\\
  \bigskip
	See the \href{https://docs.docker.com/compose/reference/}{\color{blue}\underline{Compose CLI reference}} for more.\\
  \bigskip
  Installation instructions for Docker and Compose can be found \href{https://docs.docker.com/engine/install/ubuntu/}{\color{blue}\underline{here}} and \href{https://docs.docker.com/compose/install/linux/}{\color{blue}\underline{here}}, respectively, and also in the provided shell script \href{https://github.com/IntelligentSystemsLabUTV/ros2-examples/blob/humble/bin/docker_install.sh}{\color{blue}\underline{\texttt{bin/docker\_install.sh}}}.\\
  On systems with an Nvidia GPU, the \href{https://docs.nvidia.com/datacenter/cloud-native/container-toolkit/overview.html}{\color{blue}\underline{NVIDIA Container Toolkit}} is suggested.\\
  Installation on Windows and macOS requires \href{https://www.docker.com/products/docker-desktop/}{\color{blue}\underline{Docker Desktop}}, but \textbg{lacks many features}.
\end{frame}

% --- Example ---
\begin{frame}{Example}{\texttt{ros2-examples}}
  Remember \href{https://github.com/IntelligentSystemsLabUTV/ros2-examples}{\color{blue}\underline{ros2-examples}}?\\
  It is a self-contained \textbg{ROS 2 development environment built with Docker}!\\
  \bigskip
  It also supports the \textbg{automated build} of \textbg{development containers} in \textbg{VS Code}.\\
	Such containers are based on our \href{https://github.com/IntelligentSystemsLabUTV/dua-template}{\color{blue}{\underline{\textbf{Distributed Unified Architecture}}}} project, which is part of Roberto Masocco's PhD thesis.\\
  \bigskip
	Their inner workings are totally transparent, but if you're curious see \href{https://github.com/IntelligentSystemsLabUTV/ros2-examples/blob/humble/dua_template.md}{\color{blue}\underline{\texttt{dua\_template.md}}}.
\end{frame}
